\documentclass[paper=a4, fontsize=11pt]{article}

%----------------------------------------------------------------------------------------
%	PACKAGES AND OTHER DOCUMENT CONFIGURATIONS
%----------------------------------------------------------------------------------------

\usepackage{amsmath,amsfonts,amsthm} % Math packages
\usepackage{sectsty} % Allows customizing section commands
\allsectionsfont{\centering \normalfont\scshape} % Make all sections centered, the default font and small caps

\usepackage{fancyhdr} % Custom headers and footers
\pagestyle{fancyplain} % Makes all pages in the document conform to the custom headers and footers
\fancyhead{} % No page header - if you want one, create it in the same way as the footers below
\fancyfoot[L]{} % Empty left footer
\fancyfoot[C]{} % Empty center footer
\fancyfoot[R]{\thepage} % Page numbering for right footer
\renewcommand{\headrulewidth}{0pt} % Remove header underlines
\renewcommand{\footrulewidth}{0pt} % Remove footer underlines
\setlength{\headheight}{13.6pt} % Customize the height of the header

\numberwithin{equation}{section} % Number equations within sections (i.e. 1.1, 1.2, 2.1, 2.2 instead of 1, 2, 3, 4)
\numberwithin{figure}{section} % Number figures within sections (i.e. 1.1, 1.2, 2.1, 2.2 instead of 1, 2, 3, 4)
\numberwithin{table}{section} % Number tables within sections (i.e. 1.1, 1.2, 2.1, 2.2 instead of 1, 2, 3, 4)

\setlength\parindent{0pt} % Removes all indentation from paragraphs - comment this line for an assignment with lots of text

\usepackage{graphicx}
\graphicspath{ {images/} }

\usepackage{xepersian}
\settextfont[Path=fonts/]{Vazir.ttf}

%----------------------------------------------------------------------------------------
%	TITLE SECTION
%----------------------------------------------------------------------------------------

\newcommand{\horrule}[1]{\rule{\linewidth}{#1}} % Create horizontal rule command with 1 argument of height

\title{
\normalfont \normalsize 
\includegraphics[scale=0.1]{aut}
\hspace{5cm}
\includegraphics[scale=0.1]{ceit} \\
\textsc دانشگاه صنعتی امیرکبیر \\
\textsc دانشکده مهندسی کامپیوتر و فناوری اطلاعات
\horrule{0.5pt} \\ [0.4cm] % Thin top horizontal rule
\huge پروژه‌ی نهایی طراحی پایگاه داده‌ها \\ % The assignment title
\horrule{2pt} \\ [0.5cm] % Thick bottom horizontal rule
}

\author{دکتر ممتازی}

\date{\normalsize\today} % Today's date or a custom date

\begin{document}

\maketitle % Print the title

\section{سایت پرسش و پاسخ}
\paragraph{
هدف این پروژه طراحی پایگاه‌داده یک وب‌سایت جهت پرسش و پاسخ است. افراد می‌توانند در این وب‌سایت ثبتنام کنند، سوالات خود را بپرسند و یا سوالاتی که می‌دانند جواب بدهند. افراد می‌توانند به سوالات و پاسخ‌ها امتیاز بدهند. امتیازات سوالات و پاسخ‌ها بر روی امتیاز کلی افراد تاثیر می‌گذارد و هر کران امتیازی باعث یک سطح اعتبار در کاربر می‌شود.
}

\begin{center}
\begin{tabular}{|c|c|}
    \hline
    امتیاز & اعتبار \\
    \hline
    100 & مبتدی \\
    500 & پیشرفته \\
    1000 & نیمه حرفه‌ای \\
    5000 & حرفه‌ای \\
    \hline
\end{tabular}
\end{center}

\paragraph{
هر پاسخ تایید شده توسط کاربر طراح سوال ۲۵ امتیازی برای پاسخ دهنده به ارمغان می‌آورد، هر رای مثبت به سوال یا پاسخ ۱۰ امتیاز را به نویسنده‌ی آن می‌دهد و این در حالی است که هر پاسخ نادرست ۱ امتیاز منفی در پی دارد.
کاربران برای مطرح کردن سوالات خود در مورد پاسخ‌ها یا سوالات می‌توانند روی سوال یا پاسخ مورد نظر، یادداشت بگذارند.
}

\subsection{موجودیت‌ها}
\begin{itemize}
    \item کاربر
    \begin{itemize}
        \item نام‌کاربری، رمز عبور و آدرس ایمیل
        \item امتیاز و اعتبار
    \end{itemize}
    \item سوال
    \begin{itemize}
        \item مطرح‌کننده و پاسخ‌دهندگان (توجه کنید که ممکن است به هر سوال کاربران مختلفی پاسخ دهند.)
        \item پاسخ منتخب توسط کاربر مطرح‌کننده‌ی سوال
        \item امتیاز سوال و پاسخ‌ها
        \item یادداشت‌های گذاشته شده روی سوالات یا پاسخ‌ها
    \end{itemize}
\end{itemize}

\subsection{نیازمندی‌های کاربر از سیستم}
\begin{itemize}
    \item امکان ثبتنام کاربران و تغییر مشخصات (نام کاربری و آدرس ایمیل نباید قبلا انتخاب شده باشند.)
    \item امکان طرح سوال
    \item امکان پاسخ دادن
    \item امکان ایجاد یادداشت برای سوالات و پاسخ‌ها
    \item امکان امتیازدهی به سوالات و پاسخ‌ها
    \item امکان مشخص کردن پاسخ منتخب توسط مطرح‌کننده‌ی سوال
\end{itemize}

\end{document}