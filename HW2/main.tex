\documentclass[12pt,a4paper]{article}
%در ورژن جدید زی‌پرشین برای تایپ متن‌های ریاضی، این سه بسته، حتماً باید فراخوانی شود
\usepackage{amsthm,amssymb,amsmath}

\usepackage{fontspec}

%دستوری برای وارد کردن واژه‌نامه انگلیسی به فارسی
\newcommand\persiangloss[2]{#1\dotfill\lr{#2}\\}
%بسته‌ای برای تنطیم حاشیه‌های بالا، پایین، چپ و راست صفحه
%\usepackage[top=30mm, bottom=30mm, left=30mm, right=30mm]{geometry}
%بسته‌ای برای نمایش تصاویر قرار داده شده در متن
\usepackage{graphicx}
% بسته‌ و دستوراتی برای ایجاد لینک‌های رنگی با امکان جهش
\usepackage[pagebackref=false,colorlinks,linkcolor=blue,citecolor=magenta]{hyperref}
% چنانچه قصد پرینت گرفتن نوشته خود را دارید، خط بالا را غیرفعال و  از دستور زیر استفاده کنید چون در صورت استفاده از دستور زیر‌‌، 
% لینک‌ها به رنگ سیاه ظاهر خواهند شد و برای پرینت گرفتن، مناسب‌تر خواهد بود.
%\usepackage[pagebackref=false]{hyperref}
%بسته‌ای برای ظاهر شدن «مراجع»  در فهرست مطالب
\usepackage{tocbibind}
%فراخوانی بسته زی‌پرشین و دستورات مربوط به نوع فونت‌ها
\usepackage{xepersian}
%تغییرات نخ
\usepackage[bottom]{footmisc}
\usepackage{indentfirst}


% وارد کردن دستور بالا الزامی نیست؛ چون در صورت وارد نکردن آن، فونت پیش‌فرض به صورت خودکار، فراخوانی می‌شود.
% چنانچه می‌خواهید که اعداد داخل فرمول‌ها، فارسی باشد، دستور زیر را فعال کنید
%\setdigitfont{Times New Roman}


%%%%%%%%%%%%%%%%%%%%%%%%%%%%%%%%%%%%%%%%%%%%%%%%%%%

% تعریف قلم‌های فارسی و انگلیسی برای استفاده در بعضی از قسمت‌های متن
\settextfont[Path=fonts/]{Vazir.ttf}
\setlatintextfont{Times New Roman}


\DefaultMathsDigits
%اگر فونت‌های بالا را ندارید، دو خط بالا را غیر فعال و دو خط زیر را فعال کنید
%\defpersianfont\traffic[Scale=1]{XB Roya}
%\defpersianfont\yekan[Scale=1]{XB Kayhan}
%%%%%%%%%%%%%%%%%%%%%%%%%%%%%%%%%%%%%%%%%%%%%%%%%%%
% تعریف و نحوه ظاهر شدن قضایا، لم‌ها، تعریف‌ها و ...

%%%%%%%%%%%%%%%%%%%%%%%%%%%%%%%%%%%%%%%%%%%%%%%%%%%
\begin{document}
% دستوری جهت حذف کردن شماره صفحه و سربرگ، در صورت وجود (فقط در صفحه جاری)
\thispagestyle{empty}
\vspace*{-28mm}
% نحوه درج کردن لوگوی دانشگاه
\centerline{\includegraphics[height=5cm]{logo.png}}
\begin{center}
%دستوری برای کم کردن فاصله بین لوگو و خط پایین آن
\vspace{-2mm}
{\large
اصول طراحی پایگاه داده
%دستوری برای تعیین فاصله بین دو خط
\\[2.1cm]
}

{\Large
تمرین سری دوم
\\[2cm]

دکتر ممتازی
\\[1.5cm]
\large 
موعد تحویل
\\[0.5cm]
---
}
%دستوری برای تعیین فاصله بین خطوط (نه دو خط) و تا وقتی که مقدار آن تغییر نکند، فاصله بین خطوط، همین مقدار است.
\baselineskip=1cm

{\large
پاییز ۱۳۹۸
}
\end{center}
%دستوری برای رفتن به صفحه جدید
\newpage
\baselineskip=1cm
%دستوری برای ظاهر شدن فهرست مطالب
\baselineskip=.75cm
\newpage 

\section{مقدمه}
در این پروژه قصد داریم پایگاه داده مربوط به یک فروشگاه آنلاین را طراحی و پیاده‌سازی نماییم.
در ادامه این فایل، ابتدا تعریف پروژه به طور مشروح ذکر می‌شود. در بخش بعدی موارد مطلوب در این تمرین ذکر می‌شوند.


\section{تعریف پروژه}
پروژه مورد نظر مربوط به یک فروشگاه زنجیره‌ای است که شعب مختلف دارد. هر شعبه این فروشگاه دارای یک صفحه پروفایل مخصوص به خود است که در آن لیست تمام محصولات موجود در این شعبه قرار دارد. 
\par
کاربران با ورود به این فروشگاه و انتخاب شعبه مورد نظر می‌توانند علاوه بر اطلاعات شعبه شامل نام، نام مسئول، آدرس، تصاویر مربوط به شعبه که می‌توانند به تعداد دلخواه باشند، اطلاعات تمام محصولاتی را که شعبه به فروش می‌رساند مشاهده نمایند. هر محصول، خود شامل اطلاعاتی از جمله نام، نام شرکت تولیدکننده، قیمت، تصاویر که می‌توانند به تعداد دلخواه باشند، میزان تخفیف، دسته‌بندی محصول که یکی از موارد تعریف شده در سیستم است، می‌باشد.
\par
هر کاربر می‌تواند با وارد کردن اطلاعات خود شامل نام،‌ نام خانوادگی، نام کاربری که باید در سامانه یکتا باشد، رمز عبور که باید حداقل ۶ کاراکتر داشته باشد، آدرس محل سکونت که اجباری است، آدرس محل کار در صورت تمایل، حداقل یک مورد شماره تماس و حداکثر یک تصویر پروفایل در سامانه ثبت‌نام نماید. 
\par
همین‌طور هر کاربر دارای یک مقدار اعتبار است که می‌توانند از طریق پرداخت الکترونیک مقدار موجود در آن را شارژ نمایند و در هنگام خرید، تمام یا بخشی از هزینه سفارشات را از آن و مابقی را به صورت پرداخت آنلاین، پرداخت نمایند.
\par
در هنگام ثبت سفارش، هر کاربر یک سبد خرید خواهد داشت که می‌تواند کالاهای مختلف را به آن اضافه و یا از آن حذف نماید. همین‌طور این امکان وجود خواهد داشت که کاربران از یک کالا، چندین نمونه با هم خریداری نمایند. تمام این اطلاعات شامل اقلام انتخاب شده، شعبه هر محصول و تعداد هر یک از این اقلام باید در سبد خرید قابل گزارش‌گیری باشد. هر سبد خرید پس از پرداخت هزینه سفارش، از طریق اعتبار حساب، پرداخت آنلاین یا هر دو مورد، از بین می‌رود.
\par
گزارش تمام سفارش‌های شامل اطلاعات کاربران و سفارش‌های آن‌ها باید به طور کامل ثبت شده و آماده برای گزارش‌گیری کامل باشند.
\par
هر کاربر می‌تواند نظر خود را در رابطه با هر محصول ثبت کرده و به هر محصول امتیازی بین 0 تا 5 بدهد. نظرات تمام کاربران باید به طور کامل به همراه اسم فرستنده قابل مشاهده باشد.
\par
همین‌طور میانگین امتیازهای داده‌شده به هر محصول توسط تمام کاربران باید قابل محاسبه باشد. نیازی به ثبت امتیاز هر کاربر به هر محصول وجود ندارد.
\par
توجه نمایید تمام گزارش‌های مالی باید به طور دقیق و کامل ثبت شوند.
\par
تمام سفارشات باید از زمان ثبت و پرداخت سفارش تا زمان تحویل سفارش به مشتری، قابل ردگیری باشند. کاربران باید بتوانند با وارد کردن شناسه هر سفارش، که در کل سامانه منحصر به فرد است، در هر لحظه از آخرین وضعیت سفارش خود مطلع شوند.
\par
کاربران می‌توانند با ارسال انتقادات و پیشنهادات خود به مدیر سایت، سامانه را از نقطه نظرات خود مطلع سازند. مدیر سایت هم می‌تواند با ارسال پاسخ به هر یک از این پیام‌ها، پاسخ نقد و یا شکایت افراد را به طور مستقیم بدهد. این مکالمات می‌توانند تا هر مرحله‌ دلخواهی پیش بروند و گزارش کامل تمام مکالمات باید قابل نمایش باشد.
\par
دسته‌بندی تمام محصولات باید ثبت شوند به طوری که دسته‌های مختلف در سامانه تعریف شده باشد و قابلیت اضافه کردن دسته‌بندی جدید و یا حذف و ویرایش دسته‌بندی‌های موجود به راحتی وجود داشته باشد. از تکرار بی‌مورد دسته‌بندی‌های تعریف شده در موارد غیر ضروری بپرهیزید.
\par
دقت نمایید در تمام رکوردهای موجود در تمام جداول باید حتما زمان ثبت رکورد و هم‌چنین زمان آخرین به‌روزرسانی رکورد مشخص باشد. زمان ثبت اولیه رکورد به طور خودکار در جداول درج شود.

\section{مطلوبات پروژه} 
موارد زیر را به دقت انجام داده و نتایج حاصل را در یک فایل زیپ، که مطابق با آن‌چه در سایت درس توضیح داده شده نام‌گذاری شده است، در سایت درس بارگذاری نمایید. موارد مطلوب در این پروژه به شرح زیر هستند:


\begin{enumerate}
\item لیست تمام \lr{Entity}های سیستم را استخراج و ذکر نمایید. تمامی \lr{Weak Entity}ها را به طور دقیق مشخص نمایید و بگویید  \lr{Strong Entity} مربوط به هر یک از آن‌ها کدام است.

\item لیست تمام \lr{Relation}ها را تهیه کنید.


\item لیست تمام کلید‌ها و صفات \lr{Entity}ها و \lr{Relation}ها را تهیه کنید.

\item برای هر \lr{Entity} لیست تمام \lr{Attribute}ها را شامل نام، نوع و مقادیر پیش‌فرض و \lr{Option}های دیگر مشخص نمایید. توجه کنید حتما \lr{Primary Key}ها و \lr{Foreign Key}ها به طور دقیق مشخص شده باشند. در مورد \lr{Foreign Key}ها بگویید ارجاع آن‌ها به کدام مولفه از کدام \lr{Entity} باز می‌گردد.

\item \lr{ERD} نهایی را به طور کامل رسم نمایید. روابط تمام \lr{Entity}ها و \lr{Cardinality} تمام روابط باید به طور دقیق در این نمودار مشخص شده باشد.

\item تمام دستورات لازم برای ایجاد پایگاه‌داده مطابق با طراحی انجام شده را در یک فایل متنی بنویسید به طوری‌که اجرای فایل متنی به طور کامل و بدون اشکال، پایگاه داده طراحی شده را ایجاد نمایید.

\item اکسپورت پایگاه داده ایجاد شده را ضمیمه فایل‌های ارسالی نمایید.
\end{enumerate}



\end{document} 
